% ================================================================================================
\documentclass[10pt,notitlepage]{article}

\usepackage[T1]{fontenc}
\usepackage{amsmath}
\usepackage{bm}
\usepackage[urw-garamond]{mathdesign}

% Links
\usepackage[linktocpage]{hyperref}

% To type in pt-br.
\usepackage[brazilian]{babel}
\usepackage[utf8]{inputenc}

% Page layout
\usepackage[usenames,dvipsnames]{color}
\usepackage[margin=1.0in]{geometry}
\usepackage{fancyhdr}
\pagestyle{fancy}
\fancyfoot{}
\fancyhead{}
\fancyhead[L]{Teoria de perturbação de buracos negros}
\renewcommand{\headrulewidth}{0.5pt}
\fancyfoot[C]{--~\thepage~--}

% To generate random text.
\usepackage{lipsum}

% For french-style print use:
% \usepackage[uppercase=upright, greeklowercase=upright, urw-garamond]{mathdesign}
\usepackage{empheq}
\newcommand\widecolourbox[1]{{\setlength\fboxrule{0.75pt}\setlength\fboxsep{4pt}\fcolorbox{Maroon}{white}{\enspace#1\enspace }}}

% Some user-defined commands
\newcommand{\dd}{{\rm d}}
% ================================================================================================

\begin{document}

\title{\large \sc{O formalismo de Newman-Penrose}}
\author{\small \sc{Hector O. Silva}}
\date{}
\maketitle

Considere o tensor de Einstein, definido como:
%
\begin{empheq}[box=\widecolourbox]{equation}
G_{ab} \equiv R_{ab} - \frac{1}{2} R g_{ab}\,,
\end{empheq}
%
e uma equação qualquer do livro do Poisson-Will:
%
\begin{equation}
{\cal J}^{jL} = \epsilon^{jab} \int_{\cal M} c^{-1} \tau^{0b}(t,\bm{x}) x^{a L} \, \dd^3 x\,.
\end{equation}
%
onde $L$ é um multi-índice contendo $\ell$ índices, tal que
$A^{L} \equiv A^{j_{1} j_{2} \dots j_{\ell}}$ por exemplo.
%
Citando um artigo qualquer \cite{Zerilli:1971wd}.

\lipsum[2-8]
\lipsum[2-3]

\bibliographystyle{apalike}
\bibliography{biblio}

\end{document}
